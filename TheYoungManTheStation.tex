





\documentclass[twoside,british,a4paper]{article}


\usepackage{amsmath}
\usepackage{amsfonts}
\usepackage{amssymb,upref}
\usepackage{fourier}
\usepackage{tgheros}
\usepackage[T1]{fontenc}
\usepackage{textcomp}% add latexsym if you need it
\usepackage{microtype}
\usepackage{anyfontsize}
\usepackage{xcolor}


%% Use accented characters
\usepackage[utf8]{inputenc}
\usepackage[T1]{fontenc}


\usepackage{isodate}
\cleanlookdateon

%\usepackage{showframe}
\usepackage{lipsum}



\usepackage{geometry}
\geometry{reset,ignoreall,
  textheight=240mm,
  textwidth=140mm,
  bottom=30mm,
  inner=31.5mm,
  headheight=12.4pt
  }
\setlength{\parindent}{0em}
\setlength{\parskip}{0.6\baselineskip}






\usepackage{fancyhdr}
\renewcommand{\headrule}{}
\renewcommand{\footrule}{}
%
\pagestyle{fancy}
\fancyhead{}\fancyfoot{}
\fancyhead[LE]{ \footnotesize\sffamily\upshape \shortauthor}
\fancyhead[RO]{ \footnotesize\sffamily\upshape \shorttitle}
\fancyfoot[LE,RO]{\sffamily\bfseries\upshape \thepage}
%
\fancypagestyle{firstpagestyle}
\fancyhead{}\fancyfoot{}
\fancyfoot[LE,RO]{\sffamily\bfseries\upshape \thepage}

\usepackage{titlesec}
\titleformat{\section}[block]{\large\bfseries\upshape\sffamily\boldmath}{\thesection}{0.5em}{}

\usepackage{titling}
\setlength{\droptitle}{-12mm}
\pretitle{\noindent\begin{minipage}{\textwidth}\LARGE\bfseries\upshape\sffamily\boldmath}
\posttitle{\end{minipage} \vskip 1.0em}
\preauthor{\noindent \begin{minipage}{\textwidth} \large\mdseries\sffamily}
\postauthor{\end{minipage} \vskip 0.8em}
\predate{ \noindent \hspace{-0.4em} \sffamily\small}
\postdate{\vskip 1.4em}

\renewenvironment{abstract}{
   \noindent\begin{minipage}{\textwidth}
   \upshape\sffamily \small
  }{
   \par \vskip 1.0em
   {
   \color{black!60}
   \noindent \footnotesize
   \address
   \par
   \authoremail
   }
   \end{minipage} \vskip 4em
  }



\usepackage[%
  bookmarks=true,
  colorlinks,
  linkcolor=blue,
  urlcolor=blue,
  citecolor=blue,
  plainpages=false,
  pdfpagelabels,
  final,
  breaklinks=true,
  backref=page
]{hyperref}
\hypersetup{
pdftitle={The young man the station}, 
pdfauthor={Emilio Pisanty},
pdfkeywords={latex, article, class}
}

\title{The young man the station}
\newcommand\shorttitle{short title}%{The young man the station}

\author{Emilio Pisanty\textsuperscript{1$\,*$}}
\newcommand\shortauthor{E. Pisanty}

\newcommand\address{\textsuperscript{1}ICFO -- The Institute of Photonic Sciences}
\newcommand\authoremail{$^*$emilio.pisanty@icfo.eu}

\date{ \today }











\begin{document}

\maketitle
\thispagestyle{firstpagestyle}

\begin{abstract}



The Young Man The Station is a LaTeX template built on top of the standard \textrm{article} class, and built on the notion that not all LaTeX-formatted papers need to use Computer Modern, be in two columns, or generally share the same old and tired look. The name of the template is a variation of a standard so-called `garden-path sentence' and despite appearances it is in fact grammatically correct.
\end{abstract}




\noindent
This is a paragraph with some text at the start of the article. This is just a sample but it has some mathematics like $x=y^2 +A_\mathrm{eff}$ and something else like $\hat{V}_\mathsf{L} = \sum_j \mathbf{r}_j^k$. Then it just starts repeating itself. This is a paragraph with some text at the start of the article. Then it just starts repeating itself. This is just a sample but it has some mathematics like $\chi = a_{d} \xi \zeta/\lambda = m_\mathrm{eff}d^2/(\hbar^2 \lambda)$ and something else like $r=|\vec{x}_A-\vec{x}_B|=\sqrt{(x_A-x_B)^2+(y_A-y_B)^2}$ or $d  \mathopen{}\left(  p+A(t)  \right)\mathclose{}  ^*$ . Then it just starts repeating itself.

This is another paragraph with more text just to show paragraph breaks. If I say text, text, text, do you say go, go, go? This is really just filler text but to be honest including a whole paragraph of Lore Ipsum here seemed a bit too much.






\section{Here are some maths}
\lipsum[4]

Here's an equation with a label, \eqref{first-equation} and there's more equations further down. So, we have
\begin{equation}
V(\mathbf{x}_A,\mathbf{x}_B)=V(\vec{x}_A,\vec{x}_B)=d^2\frac{r^2-2\lambda^2}{(r^2+\lambda^2)^{5/2}},
\label{first-equation}
\end{equation}
being $d$ a letter, $\lambda$ a gathingammy $r$ a letter in $r=|\vec{x}_A-\vec{x}_B|=\sqrt{(x_A-x_B)^2+(y_A-y_B)^2}$, with  $A$, $B$  labels. Moreover $\Lambda=\lambda/a$, and  $\chi = a_{d}/\lambda = m_{eff}d^2/(\hbar^2 \lambda)$, with $m_\mathrm{eff}=\hbar^2/2ta^2$, and $t$, are more maths expressions. So are $k=\sqrt{k_x^2+k_y^2}$ and $V_{latt}(\vec r)= V_0\left(\sin^2(k_x x)+ \sin^2(k_y y)\right))$, and a displayed equation is 
\begin{equation}
\left(\hat{T}_A+\hat{T}_B+{\hat V}(\vec{x}_A,\vec{x}_B)\right)\Phi(\vec{x}_A,\vec{x}_B)=E\Phi(\vec{x}_A,\vec{x}_B).
\end{equation}

Other displayed equations are
\begin{equation}
(\vec{\xi}_{\vec{K}}\cdot\vec{\hat{T}}_D+V(\vec{r}))\psi(\vec{r})=E\psi(\vec{r}),
\end{equation}
where $\vec{\xi }_{\vec{K}}=-2t(\cos(K_x a/2),\cos(K_y a/2))$ and $\vec{\hat{I}}\cdot\vec{\hat{T}}_D\psi(\vec{r})=\sum_{i=x,y}\left(\psi(\vec{r}+\vec{\delta}_i)+\psi(\vec{r}-\vec{\delta}_i)\right)$, where $\vec{\delta}_i=a\hat{e}_{i}$, and also
\begin{equation}
\psi(\vec{r})=\frac{1}{N_x N_y}\sum_{\vec{q}}\psi(\vec{q})e^{i\vec{q}\cdot\vec{r}}.
\end{equation}
Here's a little bit of text to round out the paragraph after the equation and (probably) round out the page, so this needs to be a couple of lines long.

Here some text: \lipsum[7] 

Some text, then some more equations: these ones are
$$
E_{\vec{K},\vec{q}}=-4t\left(\cos(K_xa/2)\cos(q_xa)+\cos(K_ya/2)\cos(q_ya)\right)
$$ 
and 
\begin{equation}
(E-E_{\vec{K},\vec{q}})\psi(\vec{q})=\sum_{\vec{q'}}V(\vec{q}-\vec{q'})\psi(\vec{q'}). 
\end{equation}
To round out the paper, we now include a bunch of Lorem Ipsum paragraphs, some more math, more filler text, and call it a day.

\lipsum[1-3]

Here $\vec{\xi }_{\vec{K}}=-2t(\cos(K_x a/2),\cos(K_y a/2))$ and $\vec{\hat{I}}\cdot\vec{\hat{T}}_D\psi(\vec{r})=\sum_{i=x,y}\left(\psi(\vec{r}+\vec{\delta}_i)+\psi(\vec{r}-\vec{\delta}_i)\right)$, where $\vec{\delta}_i=a\hat{e}_{i}$, and also
\begin{equation}
\psi(\vec{r})=\frac{1}{N_x N_y}\sum_{\vec{q}}\psi(\vec{q})e^{i\vec{q}\cdot\vec{r}}
\end{equation}
and
$$
E_{\vec{K},\vec{q}}=-4t\left(\cos(K_xa/2)\cos(q_xa)+\cos(K_ya/2)\cos(q_ya)\right)
$$ 
and 
\begin{equation}
(E-E_{\vec{K},\vec{q}})\psi(\vec{q})=\sum_{\vec{q'}}V(\vec{q}-\vec{q'})\psi(\vec{q'}). 
\end{equation}


\lipsum[1-6]




\end{document}
